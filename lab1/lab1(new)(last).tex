\documentclass[11pt]{article}
\usepackage{amsmath,amssymb,amsthm}
\usepackage{algorithm}
\usepackage[noend]{algpseudocode} 
\usepackage{titlesec}
\usepackage{enumitem}
\usepackage[T2A,T1]{fontenc}
\usepackage{amsmath,amssymb,amsthm}
\usepackage{fancyhdr}
\usepackage{indentfirst}

%---enable russian----

\usepackage[utf8]{inputenc}
\usepackage[russian]{babel}

% PROBABILITY SYMBOLS
\newcommand*\PROB\Pr 
\DeclareMathOperator*{\EXPECT}{\mathbb{E}}


% Sets, Rngs, ets 
\newcommand{\N}{{{\mathbb N}}}
\newcommand{\Z}{{{\mathbb Z}}}
\newcommand{\R}{{{\mathbb R}}}
\newcommand{\Zp}{\ints_p} % Integers modulo p
\newcommand{\Zq}{\ints_q} % Integers modulo q
\newcommand{\Zn}{\ints_N} % Integers modulo N

% Landau 
\newcommand{\bigO}{\mathcal{O}}
\newcommand*{\OLandau}{\bigO}
\newcommand*{\WLandau}{\Omega}
\newcommand*{\xOLandau}{\widetilde{\OLandau}}
\newcommand*{\xWLandau}{\widetilde{\WLandau}}
\newcommand*{\TLandau}{\Theta}
\newcommand*{\xTLandau}{\widetilde{\TLandau}}
\newcommand{\smallo}{o} %technically, an omicron
\newcommand{\softO}{\widetilde{\bigO}}
\newcommand{\wLandau}{\omega}
\newcommand{\negl}{\mathrm{negl}} 

% Misc
\newcommand{\eps}{\varepsilon}
\newcommand{\inprod}[1]{\left\langle #1 \right\rangle}



\newcommand{\handout}[5]{
	\noindent
	\begin{center}
		\framebox{
			\vbox{
				\hbox to 5.78in { {\bf Научно-исследовательская практика} \hfill #2 }
				\vspace{4mm}
				\hbox to 5.78in { {\Large \hfill #5  \hfill} }
				\vspace{2mm}
				\hbox to 5.78in { {\em #3 \hfill #4} }
			}
		}
	\end{center}
	\vspace*{4mm}
}

\newcommand{\lecture}[4]{\handout{#1}{#2}{#3}{Scribe: #4}{Система верстки LaTeX}}

\newtheorem{theorem}{Теорема}
\newtheorem{definition}{Определение}

\begin{document}
	
	\lecture{}{Лето 2020}{}{Косберг Андрей}
	\newpage
	\renewcommand{\headrulewidth}{0pt}
	
	
	огромная и громоздкая структура, разделенная на большое количество областей, в которых только специалист знал свой путь. Гаусс был последним математиком, и не будет преувеличением сказать, что он был в какой-то степени связан почти со всеми аспектами предмета. Современники считали его Принцепсом Математическим (Принц Математик) наравне с Архимедом и Исааком Ньютоном. Это становится понятно в небольшом инциденте: на вопрос, кто был величайшим математиком в Германии, Лаплас ответил: «Почему, Пфафф». Когда спрашивающий указал, что он думал бы, что Гаусс был, Лаплас ответил, «Pfaff - безусловно самый большой в Германии, но Гаусс - самый большой во всей Европе».\\
	Хотя Гаусс украшал каждую отрасль математики, он всегда относился к теории чисел с большим уважением и любовью. Он настаивал на том, что «Математика - королева наук, а теория чисел - королева математики».
	
	
	
	
	
	
	
	
	\section{Основные свойства конгруэнтности}
	\noindent
	В первой главе \textit{Арифметических исследований} Гаусс вводит понятие конгруэнтности и обозначения, делающие её таким мощным методом (он объясняет, что его заставили принять символ $\equiv$ из-за тесной аналогии с алгебраическим равенством). Согласно Гауссу, «если число n измеряет разницу между двумя числами $a$ и $b$, то $a$ и $b$ называются конгруэнтными по отношению к $n$; если нет, неконгруэнтными». Подставляя это в форму определения, мы имеем
	
	\begin{definition}
		Пусть $n$ будет фиксированным положительным целым числом. два целых числа $a$ и $b$ называются конгруэнтными по модулю $n$ и обозначаются как	\[\textit{a}\equiv\textit{b}\pmod{n}\]
		если \textit{n} делится на разницу $a$ и $b$; тогда $a-b = kn$ для некоторого целого числа $k$.
		\\
		\\
		Чтобы закрепить идею, рассмотрим $n = 7$. Это обычная проверка
		\[
		3\equiv24 \pmod{7}, -31\equiv11 \pmod{7}, -15\equiv-64 \pmod{7}\]
		\[	since 3-24=(-3)7, -31-11=(-6)7, and -15-(-64)=7\cdot7.
		\]
	\end{definition}
			
		
	
	Если $N$ (не понял, что за символ) $(a-b)$, то мы говорим, что a не соответствует $b$ по модулю $n$, и в этом случае мы пишем $a \not\equiv b$ $\pmod{n}$. Например: $25 \not\equiv 12 \pmod{7}$, так как $7$ не удается разделить $25-12 = 13$.
	
	Следует отметить, что любые два целых числа являются конгруэнтными по модулю $1$, тогда как два целых числа являются конгруэнтными по модулю $2$, когда они оба четные или оба нечетные. Поскольку сравнение по модулю $1$ не представляет особого интереса, обычная практика состоит в предположении, что $n> 1$.
	
	Учитывая целое число $a$, пусть $q$ и $r$ будут его частным и остатком при делении на $n$, так что
	
	\[
		a=qn+r,  0 \leq r<n.
	\]
	
	Тогда по определени конгруэнции a $\equiv$ r $\pmod{n}$. поскольку существует $r$ вариантов для $r$, мы видим, что каждое целое число конгруэнтно по модулю $n$ ровно одному из значений $0,1,2 ..., n-1$; в частности, $a \equiv 0 \pmod{n}$ тогда и только тогда, когда $n | a$. Множество $n$ целых чисел $0, 1, 2, ..., n-1$ называется множеством наименьших положительных вычетов по модулю $n$.
	
	Обычно говорят, что набор из n целых чисел $a_{1}, a_{2}, ..., a_{n}$ образует полный набор вычетов (или полную систему вычетов) по модулю $n$, если каждое целое число конгруэнтно по модулю $n$ одному и только одному из $a_{k}$; иными словами, $a_{1}, a_{2}, ..., a_{n}$ конгруэнтны по модулю $n$ с $0,1,2, ..., n-1$, взятыми в некотором порядке. Например,
	\[
		-12, -4,11,13,22,82,91
	\]
	
	составляют полный набор остатков по модулю $7$; здесь мы имеем
	\[
		-12 \equiv 2, -4 \equiv 3, 11 \equiv 4, 13 \equiv 6, 22 \equiv 1,82 \equiv 5,91 \equiv 0
	\]
	все по модулю $7$. Наблюдение некоторой важности состоит в том, что любые $n$ целых чисел образуют полный набор вычетов по модулю $n$ тогда и только тогда, когда никакие два целых числа не являются конгруэнтными по модулю $n$. Нам понадобится этот факт позже.
	
	Наша первая теорема дает полезную характеристику конгруэнтности по модулю $n$ в терминах остатков при делении на $n$.


	\begin{theorem}
		Для произвольных целых чисел $a$ и $b$ $a = b \pmod{n}$ тогда и только тогда, когда $a$ и $b$ оставляют один и тот же неотрицательный остаток при делении на $n$.
	\end{theorem}
	\begin{proof}
		сначала возьмем $a = b \pmod{n}$, так что $a-b + kn$ для некоторого целого числа $k$. При делении на $n$ $b$ оставляет определенный остаток $r: b = qn + r$, где $0 \leq r <n$. Следовательно,
		\[
		а = b + kn = (Qn + r) + kn = (d + k) n + r,
		\]
		который указывает, что $а$ имеет тот же остаток, что и $b$.
		С другой стороны, предположим, что мы можем записать $a = q_{1}n + r и b = q_{2}n + r$ с тем же остатком $r (0 \leq r <n)$. затем
		\[
		a-b = (q_{1}n + r) - (q_{2}n + r) = (q_{1}-q_{2}) n,
		\]
		откуда $n | a-b$. На языке сравнений это говорит о том, что $a \equiv b \pmod{n}$.
	\end{proof}
	
	
	
	\section{Пример 4-1}
	Поскольку целые числа $-56$ и $-11$ можно выразить в виде
	\[
		-56 = (- 7) 9 + 7, -11 = (- 2) 9 + 7
	\]
	с тем же остатком $7$ теорема $4-1$ говорит нам, что $-56 = -11 \pmod{9}$. В другом направлении конгруэнтность $-31 \equiv 11 \pmod{7}$ подразумевает, что $-31$ и $11$ имеют одинаковый остаток при делении на $7$; это ясно из реалий
	\[
		-31 = (-5) 7 + 4, 11 = 1 \cdot 7 + 4.
	\]
	Конгруэнтность можно рассматривать как обобщенную форму равенства в том смысле, что ее поведение в отношении сложения и умножения напоминает обычное равенство. Некоторые из элементарных свойств равенства, которые переносятся на конгруэнции, появляются в следующей теореме.
	
	\begin{theorem}
		Пусть $n> 0$ фиксировано и $a, b, c, d$ - произвольные целые числа. Тогда выполняются следующие свойства:
	\end{theorem}
	
	\begin{enumerate}
		\item $a \equiv a \pmod{n}$
		\item Если $a \equiv b \pmod{n}$, то $b \equiv a \pmod{n}$.
		\item Если $a \equiv b \pmod{n}$ и $b \equiv c \pmod{n}$, то $a \equiv c \pmod{n}$
		\item Если $a \equiv b \pmod{n}$ и $c \equiv d \pmod{n}$, то $a + c \equiv b + d \pmod{n}$ и $ac \equiv bd \pmod{n}$.
		\item Если $a \equiv b \pmod{n}$, то $a + c \equiv b + c \pmod{n}$ и $ac \equiv bc \pmod{n}$.
		\item Если $a \equiv b \pmod{n}$, то $a^{k} \equiv b^{k} \pmod{n}$ для любого натурального числа $k$.
	\end{enumerate}
	\begin{proof}
		 для любого целого числа $a$ мы имеем $a-a = 0 \cdot n$, так что $a \equiv a \pmod{n}$. Теперь, если $a \equiv b \pmod{n}$, то $a-b = kn$ для некоторого целого числа $k$. Следовательно, $b-a = - (kn) = (- k) n$ и, поскольку $-k$ является целым числом, это дает (2).
	\end{proof}
	
	Свойство (3) немного менее очевидно: предположим, что $a \equiv b \pmod{n}$ и $b \equiv c \pmod{n}$. Тогда существуют целые числа $h$ и $k$, обозначающие $a-b = hn$ и $b-c = kn$. Это следует из того
	\[
		а-с = (а-b) + (b-с) = Hn + kn = (h + к) n,
	\]
	
	вследствие чего $a \equiv c \pmod{n}$
	
	В том же духе, если $a \equiv b \pmod{n}$ и $c \equiv d \pmod{n}$, то мы уверены, что $a-b = k_{1}n и c-d = k_{2}n$ для некоторого выбора $k_{1} и k_{2}. $Сложив эти уравнения, получим
	
	\[
		(a + b) - (b + d) = (a-b) + (с-d) -k_{1}n + k_{2}n- (k_{1} + k_{2}) n
	\]
	или, как конгруэнтное утверждение, $a + c  \equiv b + d \pmod{n}$. Что касается второго утверждения (4), отметим, что
	\[
		 ac = (b + k_{1} n) (d + k_{2}n) = ba + (bk_{2} + dk_{1} + k_{1}k_{2}n) n.
	\]
	Поскольку $bk_{2} + dk_{1} + k_{1}k_2n$ является целым числом, это говорит о том, что $ac-bd$ делится на $n$, откуда $ac \equiv bd \pmod{n}$.
	
	Доказательство свойства (5) охватывается (4) и тем, что c $\equiv$ c $\pmod{n}$. Наконец, мы получаем (6), приводя аргумент индукции. Безусловно, утверждение верно для $k = 1$, и мы будем предполагать, что оно верно для некоторого фиксированного $k$. Из (4) мы знаем, что $a \equiv b \pmod{n} и a^{k} \equiv b^{k} \pmod{n}$ вместе означают, что $aa^{k} \equiv bb^{k} \pmod{n}$ или, что эквивалентно, $a^{k} + 1 \equiv b^{k} + 1 \pmod{n}$. Это та форма, которую оператор должен принять для $k + 1$, поэтому шаг индукции завершен.
	
	Прежде чем идти дальше, мы должны проиллюстрировать большую помощь, которую могут оказать сравнения при выполнении определенных типов вычислений.
	\section{Пример 4-2}
	Попробуем показать, что $41$ делит $2^{20} -1$. Мы начинаем с нуля, что $2^{5} \equiv - 9$ $\pmod{41}$, откуда $(2^{5})^{4} \equiv (- 9)^{4}$ $\pmod{41}$ по теореме $4-2$ (6); другими словами,$ 2^{20} \equiv 81 \cdot 81 \pmod{41}. Ho 81 \equiv - 1 \pmod{41} $и так $81 \cdot 81 \equiv 1 (mod 41).$ Используя части (2) и (5) теоремы $4-2$, мы, наконец, приходим к
	
	\[
		2^{20}-1 \equiv 81 \cdot 81-1 \equiv 1-1 \equiv 0 \pmod{41}.
	\]
	Таким образом, $41 | 2^{20}-1$, по желанию.
	\section{Пример 4-3}
	Для другого примера в том же духе предположим, что нас просят найти остаток, полученный при делении суммы
	\[=
		1! +2! +3! +4! + ... + 99! +100!
	\]
	на 12. Без помощи сравнений это был бы потрясающий расчет. Наблюдение, которое начинает нас с того, что $4! \equiv 24 \equiv 0 \pmod{12}$ ; таким образом, для $k\geq 4$,
	\[
		k! = 4! \cdot 5 \cdot 6 ... k \equiv 0 \cdot 5 \cdot 6 ... k \equiv 0
		\pmod{12}	\]
		
		

	
	
	
\end{document}